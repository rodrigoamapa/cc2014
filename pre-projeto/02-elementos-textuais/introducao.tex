%
% Documento: Introdução
%

\chapter{Introdução}\label{chap:introducao}  

O presente trabalho aborda o contexto dos sistemas de georreferenciamento de tempo real, que, usualmente necessitam de abordagens escaláveis para o gerenciamento e o armazenamento de grandes quantidades de dados, mas que tradicionalmente são implementados para funcionarem sobre sistemas de bancos de dados relacionais, que dispõem de pouca escalabilidade para armazenar e gerenciar excessivos volumes de dados produzidos em baixa latência de tempo. Para contornar essa situação, esta pesquisa visa apresentar uma solução viável para essa problemática, onde ela será utilizada para a geolocalização de viatura da Guarda Civil Municipal de Macapá. Nesse trabalho, os aspectos essenciais da pesquisa proposta são apresentados e discutidos, tais como: a problemática a ser solucionada, os objetivos e justificativas, a metodologia e o cronograma de atividades a serem realizadas e desenvolvidas. Ao final desse capítulo introdutório, também é apresentada a organização do trabalho proposto.

\section {Contexto}

De acordo com uma pesquisa realizada pelo IBGE, cada vez mais residências no Brasil têm computadores ou outros dispositivos conectados à internet\cite{IBGE-2018}. Isto mostra que as informações digitais estão cada vez mais presentes no cotidiano da sociedade e não apenas nos negócios.

Para Kevin Ashton, as pessoas costumavam guardar seus objetos em gavetas de armários, dentro de cômodas, cofres e até debaixo da cama, depois da digitalização deles, passaram a guardar em banco de dados \cite{Ashton-2016}.

Os Banco de Dados foram criados para guardar informações. Elas podem ser de diferentes tamanhos e formatos que ficam armazenados em espaços organizados e categorizados\cite{Kotaro-2005}.

A organização dos Dados dentro de um Banco de Dados pode variar, mas geralmente, possui quatro operações básicas, dentre elas: adicionar, remover, copiar e atualizar os dados. Essas operações geralmente são realizadas por um software que gerencia e controla todo o fluxo que se passa em um Banco de Dados. Esses softwares são chamados de Sistemas de Gerenciamento de Banco de Dados -- SGBD\cite{C.Date}.

Inicialmente os dados computacionais eram armazenados em arquivos\cite{Kotaro-2005}, o que dificultava seu gerenciamento porque todas as informações eram postas de forma desorganizada e não categorizada, fazendo com que uma busca, por exemplo, fosse uma tarefa complicada. Além das informações estarem desorganizadas, ainda havia o problema da confiabilidade e a durabilidade dos dados armazenados, comprometendo, de forma geral, a segurança das informações, pois quem tivesse acesso ao computador onde os dados fossem armazenados, poderia manipular o arquivo, comprometendo tudo e, ainda, qualquer pane que o computador apresentasse iria influenciar no registro de informações\cite{Kotaro-2005}.

Posteriormente, as informações foram melhor organizadas em colunas e linhas, formando tabelas, conhecidas como relações, o que permitiu muitas soluções a problemas de armazenamento para as corporações e para os sistemas gerenciadores de banco de dados\cite{C.Date}.

Durante a década de 1960 vários estudos foram realizados para formação de diferentes modelos de Banco de Dados, como o Hierárquico e o de Redes. Entretanto, foi a partir da década seguinte, em 1970, com a publicação do artigo \textit{A Relational Model of Data for Large Shared Data Banks}, de Edgar Frank Codd, que foi apresentado o modelo Relacional\cite{C.Date}.

O modelo relacional surgiu para aumentar a independência dos Dados nos SGBDs e para compor uma estrutura mais sólida e durável para o armazenamento e recuperação dos dados \cite{Kotaro-2005}. Com esse modelo, algumas características que outros modelos não possuíam surgiram, como não ter caminhos pré-definidos para se fazer acesso aos dados; implementar estruturas de dados organizadas em relações e; restrição de repetição de informações \cite{Kotaro-2005}.

Com a popularização e aperfeiçoamento do modelo relacional e utilização da linguagem SQL pelos Sistemas de Gerenciamento de Banco de Dados, por muitos anos as corporações e até mesmo donos de sites utilizaram os Banco de Dados Relacionais como armazenamento de suas informações\cite{Taurion-big-data}. Como os dispositivos conectados a internet passaram a ser domésticos, o fluxo de dados aumentou exponencialmente, exigindo tanto uma maior capacidade e velocidade da Internet quanto do armazenamento digital\cite{Taurion-big-data}.

Tendo em vista número crescente de dispositivos e pessoas conectados à internet \cite{IBGE-2018}, muitos órgãos governamentais e não-governamentais têm utilizado sistemas informatizados para obter informações relevantes sobre dados dos mais variados tipos para que se possam obter, a partir desses dados, novas informações específicas, como estatísticas, variações de velocidade, taxas de consumo, mapeamento de regiões, dentre outros que podem estar em um universo comumente conhecido como Internet das Coisas, do termo original, em inglês, \textit{Internet Of Things}\cite{Magrani-2018}.

Muitos sistemas de georreferenciamento de tempo real utilizam os conceitos de \textit{"Internet of Things"}, de forma a possibilitar que aparelhos eletrônicos compartilhem dados de localização \cite{Vega}. Esse compartilhamento de dados georreferenciados são utilizados em diversos domínios de aplicações, que vão desde as situações de desastes naturais, onde usuários envolvidos em operações de proteção civil necessitam de graus eficientes e eficazes de coordenação \cite{Barroso}; até para medição remota, armazenamento, processamento e visualização de informações químicas \cite{Vega}.

O pesquisador Cézar Taurion da Universidade de São Paulo explica que todos os dados gerados por grandes empresas, redes sociais, sensores e dispositivos de diferentes formas formam volumes que ultrapassam a unidade dos zetabytes ou 10(elevado a 21) bytes\cite{Taurion-big-data}, compondo o que é conhecido como \textit{Big Data}.

Taurion ainda apresenta dois conceitos de \textit{Big Data}. O primeiro se refere a um conjunto de dados cujo crescimento é exponencial e cuja dimensão está além da habilidade das ferramentas típicas de capturar, gerenciar e analisar dados\cite{Taurion-big-data}. O segundo diz que \textit{Big Data} é o termo usado pelo mercado para descrever problemas no gerenciamento de processamento de informações em grandes volumes as quais ultrapassam a capacidade das tecnologias de informações tradicionais ao longo de uma ou mais dimensões\cite{Taurion-big-data}.

Como os dados têm aumentado de forma explosiva devido ao desenvolvimento de redes sociais e computação em nuvem, tem havido um novo desafio para armazenar, processar e analisar um grande volume de dados. As tecnologias tradicionais não se tornam uma solução adequada para processar big data, de modo que uma plataforma de big data começou a surgir\cite{Park-Big-Data}. 

Para esses problemas citados, tecnologias foram desenvolvidas especificamente para tratar grandes volumes de informações e diversificadas fontes de geração de tais dados. Uma solução seria a adoção da linguagem NoSQL (Not Only SQL), onde a construção das tabelas é diferente da tradicional e formada para os mais variados tipos; Outro método adotado seria o uso dos Banco de Dados Não Relacionais; Uma outra abordagem seria os Sistemas em Nuvem\cite{Taurion-big-data}.
 
\section{Problemática}

(Sistemas de georeferenciamento armazenam grandes volumes de dados, os quais os bancos de dados não relacionais não conseguem suportar).


Como este trabalho conterá experimentação de aplicações que envolvem Banco de Dados, \textit{Big data} e Georreferenciamento é importante destacar que:

Na sociedade atual são crescentes os dispositivos que se conectam à internet e dispõem informações que podem ser usadas e podem ser aproveitadas para diversas aplicações computacionais\cite{Vega}. Dentre elas existem as que apresentam informações e cálculos de lugares, distâncias, clima, altura, velocidade, em que é muito comum o desenvolvimento de Sistemas de Informações Geográficos (SIG’s) \cite{Vega}. Esses sistemas são capazes de armazenar dados cartográficos, censitários, cadastros urbano e rural e imagens de satélite, dispondo de mecanismos para tratar as informações, assim como para consultar, recuperar, visualizar e plotar o material na base de dados georreferenciados\cite{Vega}.

Geralmente os dados gerados por Sistemas de Informações Geográficas ou Geoespaciais são dinâmicos, pois para se obter precisão é necessário que esses dados sejam atualizados em tempo real, acumulando um grande volume de dados para serem armazenados em um banco de dados do tipo relacional\cite{Prikh}.

Por esse motivo, faz-se necessário experimentar outro tipo de banco de dados, em que seja suportado um grande fluxo, com maior escalabilidade e tempo de resposta na transição de informações georreferenciais.

\section{Objetivos}
Os objetivos do presente trabalho de pesquisa são apresentados em torno de um único objetivo geral e três objetivos específicos.

\subsection{Objetivo Geral}

Avaliar o sistema de bancos de dados não relacional orientado a coluna HBase no armazenamento e no gerenciamento de volumes de dados georreferenciados de tempo real, no contexto da geolocalização de viaturas da Guarda Municipal de Macapá.

\subsection {Objetivos Específicos}

O presente trabalho de pesquisa tem os seguintes objetivos específicos:

\begin{itemize}
	\item Estudar o contexto dos sistemas de geolocalização em tempo real.
	\item Avaliar sistemas de banco de dados escaláveis para grandes volumes de dados.
	\item Propor um estudo de caso para avaliar e comparar banco de dados relacionais e banco de dados não-relacional HBase para armazenarem e gerenciarem excessivos volumes de dados georreferenciados de tempo real, produzidos em baixa latência.
	 
\end{itemize}

\section {Hipótese}

HBase é um sistema de banco de dados não relacional orientado a coluna, capaz de fornecer escalabilidade ilimitada e vertical; e mecanismos eficientes de gerenciamento de grandes volumes de dados georreferenciados de tempo real.

\section {Justificativa}
Muitos estudiosos no Brasil têm dado destaque à Segurança Pública como um dos pilares da Ordem Social. Mais especificamente, em Macapá, de acordo com dados do Fórum Brasileiro de Segurança Pública \cite{forumdeseguranca:2018}, a capital amapaense possui um dos piores cenários com relação à Segurança e os órgãos públicos envolvidos necessitam cada vez mais de investimentos para mudar esse cenário. Um dos investimentos que se pode fazer é na tecnologia computacional para se obter estratégias preventivas no combate ao crime das áreas urbanas.

Isso porque, de acordo com o Instituto Brasileiro de Geografia e Estatística, três em cada quatro domicílios do país possuem acesso à internet\cite{IBGE-2018}.

Dentre as diversas ferramentas tecnológicas desenvolvidas para auxiliar no combate da criminalidade, a Internet Of Things -- IOT, surge com uma grande diversidade de opções que podem ser adotadas pelos agentes de Segurança Pública, principalmente no que se refere a registro de informações em tempo real e tomada de decisões com base em sistemas ágeis.

Aliada a internet das coisas, o registro de informações em banco de dados que possam obter um maior desempenho com um grande fluxo de dados em tempo real pode ser desempenhado com a tecnologia da Apache HBase, pois:

O Apache HBase é um banco de dados de código aberto NoSQL que fornece acesso de leitura/gravação em tempo real a esses grandes conjuntos de dados\cite{CETAX}

Dentre metodologia a ser aplicada e as respostas adquiridas na coleta de dados, pode-se apresentar uma possível ferramenta na contribuição dos esforços das forças de Segurança Pública para o combate à criminalidade e resolução de conflitos nas áreas urbanas de Macapá.
 
Pelo motivo supracitado, apresenta-se no presente trabalho uma abordagem computacional, utilizando tecnologias correlacionadas de Internet Das Coisas, Georreferenciamento e \textit{Big Data} a fim de experimentar ferramentas digitais para uso das forças de Segurança Pública, no caso aqui abordado, a Guarda Civil Municipal de Macapá.

\section {Metodologia}
Para o desenvolvimento e experimentação do trabalho proposto serão utilizados procedimentos metodológicos divididos em duas etapas. O referencial bibliográfico e a experimentação científica.

O referencial bibliográfico destaca não somente a pesquisa textual, mas a base para argumentação, além do desenvolvimento dos códigos a serem usados nos aplicativos dos testes computacionais.

Para isto será utilizada a revisão de literatura \textit{Ad Hoc} que é um método em que a revisão é feita de forma livre, sem o uso de guias ou fases a serem seguidas\cite{Jacobsen-metodologia}.

Já a experimentação científica tem como proposta os testes e a coleta de dados para que depois sejam levantadas conclusões com base em análises objetivas ou subjetivas feitas por meio de números ou percentuais.

Desta forma o método a ser usado será o Estudo de Caso, em que um determinado fenômeno será analisado, em seu ambiente e curso naturais depois da utilização de um software\cite{Jacobsen-metodologia}.

E ainda, vale destacar a utilização do método de "Metanálise", pois a partir dele é possível fazer estudos que integram resultados sobre determinada questão pesquisada, combinando e fazendo comparativos entre eles\cite{Jacobsen-metodologia}. Esse método será usado, principalmente no comparativo dos Sistemas de Gerenciamento de Banco de Dados (SGBDs), em que diferentes softwares serão testados, como \textit{MySQL}, \textit{PostgreeSQL} e a plataforma \textit{Apache Hadoop}. 

<<<<<<<<<<<<<<<<>>>>>>>>>>>>>>>>>>>>>>
Esta parte aqui debaixo será analisada pelo professor Thiago, pois pode não ser necessária a inclusão dela neste momento.

Como base para o trabalho foi utilizado o artigo, da empresa IBM\cite{IBM}, que é um tutorial para o desenvolvimento de um aplicativo para celular. Tal aplicação usa o GPS do aparelho para enviar informações de geolocalização (latitudes e longitudes) para um ambiente de nuvem da própria IBM.

Esses dados podem ser consumidos por outra ferramenta digital, neste caso, seguindo o artigo de tutorial, os dados foram aproveitados por um sistema \textit{web} escrito em linguagem PHP, usando o protocolo MQTT, para intercomunicação entre as aplicações do hardware do celular, da nuvem e do computador de visualização. Esse esquema pode ser melhor visualizado na Lista de Figuras.

Maiores detalhes desta abordagem serão apresentados nos próximos capítulos.

Vale salientar que, embora haja neste trabalho uma contribuição de escrita no código e modificações nos aplicativos seguidos no tutorial da IBM, o objetivo central aqui proposto não está focado no desenvolvimento da ferramenta, mas na utilidade de sua aplicabilidade. Sendo assim, os resultados a serem alcançados tentarão satisfazer as hipóteses levantadas. 
