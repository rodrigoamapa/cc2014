%
% Documento: Resumo (Português)
%

\begin{resumo}
	
Aplicações de georreferenciamento em tempo real usualmente armazenam e gerenciam grandes quantidades de dados produzidos em baixa latência de tempo. Essa situação é desafiadora quando essas aplicações são orientadas a banco de dados relacionais, que possuem baixa escalabilidade para armazenarem e gerenciarem excessivos volumes de dados voláteis e não-voláteis. Este trabalho propõe uma alternativa para esse contexto. 	
	
\textbf{Palavras-chave}: Georreferenciamento. Escalabilidade. Banco de Dados.

\end{resumo}
